% Created 2019-12-05 Thu 23:41
% Intended LaTeX compiler: pdflatex
\documentclass[11pt]{article}
\usepackage[utf8]{inputenc}
\usepackage[T1]{fontenc}
\usepackage{graphicx}
\usepackage{grffile}
\usepackage{longtable}
\usepackage{wrapfig}
\usepackage{rotating}
\usepackage[normalem]{ulem}
\usepackage{amsmath}
\usepackage{textcomp}
\usepackage{amssymb}
\usepackage{capt-of}
\usepackage{hyperref}
\author{Chetan Mistry}
\date{\today}
\title{}
\hypersetup{
 pdfauthor={Chetan Mistry},
 pdftitle={},
 pdfkeywords={},
 pdfsubject={},
 pdfcreator={Emacs 26.3 (Org mode 9.1.9)},
 pdflang={English}}
\begin{document}

\tableofcontents

Slide 1: Hi, welcome to the presentation on applying deep learning to prefetching

Slide 2: CPU's are designed so that they reduce latency as much as possible
         However, memory remains to be a significant bottleneck. So much so, that
         most modern applications spend about 50\% of total cycles are spent waiting
         for data from memory.

Slide 3: Prefetching is the mechanism that loads data from an address in memory
         before it is needed, so that when it is needed by an instruction,
         it is already in the cache.
         This means that it can reduce the bottlenecks caused by memory operations
         We can improve a prefetcher by using an RNN to predict addresses

Slide 4: One particular method that a prefetcher loads an address is by
         looking at the stride, or delta, of recent memory addresses.
         For example, accesses to addresses 0, 4, 8, 12,\ldots{} seem to be
         following a delta of 4, and so the prefetcher would load in addresses
         of multiples of 4 + offset.
         A different type of prefetcher is a correlation prefetcher. This tries
         to learn patterns in memory address accesses that aren't linear.
         These prefetchers are require large tables to store address access
         information which reduce performance and so they aren't used in
         modern multicore processors

Slide 5: Typically, memory address accesses are few and far between, this means
         that, when trying to predict an address, there are a large amount that
         should and often aren't accessed, having these as possibilities however,
         makes predicting addresses difficult.
         Neural Nets work best when data is normalised, however, if addresses are
         normalised, then a significant amount of data is lost.
         We can treat the problem of predicting an address as classifying the
         next word in a sequence. This means that the address space would be a
         vocabulary.

Slide 6: When executing a program, the addresses that the program uses may change
         with every execution. This can be caused due to factors such as
         Address Space Layout Randomisation.
         However, the behaviour of the execution doesn't change, and so,
         it is possible to learn the deltas for a program whilst not being able
         to predict the raw address itself.
         As a result, deltas are used as input rather than raw addresses

Slide 7: The embedded LSTM model is where each instruction that accesses memory
         has some data about it concatenated with the corresponding delta of that
         instruction. This then gets passed in as input to the LSTM
         The alternative model is to cluster instructions based on memory address
         locations, and then generating in-cluster deltas to pass in as input
         to the LSTM

Slide 8: The embedded LSTM only models the most frequently occurring deltas, this
         helps keep the memory footprint low.
         The input deltas that are used are ones that occur 10 or more times in
         the dataset.
         The top 10 predictions made by this LSTM at this timestamp are loaded
         into cache.
         This model improves by having a larger vocabulary to work with,
         however this then incurs the fee of having a larger memory footprint

Slide 9: Clustering + LSTM work on the fact that data structures arrays and structs
         are stored in contiguous blocks of data and accessed repeatedly
         This model performs prefetching in a local context by using the deltas
         of instructions which access similar addresses.
         The clustering is done by K-Means and the Deltas are computed within
         each cluster.
         This model implements the Adagrad Algorithm

Slide 10: Adagrad is an adaptive gradient descent algorithm that has a per-parameter
          learning rate, which means that at every timestep, the weights for every
          parameter gets updated individually.
          The adaptation that Adagrad implements is that parameters that occur
          frequently have their learning rates reduced, whereas less frequently
          occurring parameters have a higher learning rate.

Slide 11: At each timestep, 10 predictions are made. If any of the deltas are
          correct, then that set of predictions is deemed to be correct.
          If the actual delta is outside of the model's vocabulary, then the
          set is an instant failure.
          The figure shows the performance of the embedded and clustering LSTM
          models against 2 other prefetchers.
          From this we can see that the 2 prefetchers that use the LSTM
          perform at least as good as the other prefetchers.

Slide 12: Thanks
\end{document}
