% Created 2019-11-07 Thu 12:41
% Intended LaTeX compiler: pdflatex
\documentclass[11pt]{article}
\usepackage[utf8]{inputenc}
\usepackage[T1]{fontenc}
\usepackage{graphicx}
\usepackage{grffile}
\usepackage{longtable}
\usepackage{wrapfig}
\usepackage{rotating}
\usepackage[normalem]{ulem}
\usepackage{amsmath}
\usepackage{textcomp}
\usepackage{amssymb}
\usepackage{capt-of}
\usepackage{hyperref}
\author{Chetan Mistry}
\date{\today}
\title{String Format Attack}
\hypersetup{
 pdfauthor={Chetan Mistry},
 pdftitle={String Format Attack},
 pdfkeywords={},
 pdfsubject={},
 pdfcreator={Emacs 26.3 (Org mode 9.1.9)},
 pdflang={English}}
\begin{document}

\maketitle
\tableofcontents


\section{Abstract}
\label{sec:org39fae9d}

A string format attack works by exploiting vulnerabilities in source
code to examine and/or modify the values stored in variables.


\section{String Formatting in C}
\label{sec:org9ec0dda}

Some String manipulation functions in C have vulnerabilities which can
be exploited by an attacker. An example of this is the function
\texttt{scanf()} which allows the user to provide an arbitrary input.

\subsection{\texttt{scanf()}}
\label{sec:org6b6c3f6}

\texttt{scanf()} is a function which reads from stdin and stores it into a
variable. This input can be anything (strings, numbers, etc), and can
have any length.

\subsection{\texttt{printf()}}
\label{sec:org2b33583}

\texttt{printf()} is a function which takes in a string and any parameters to
that string (e.g variables) and will output to stdout. The variables
can be printed by using special tokens in the string which will map to
specific output formats, for example \texttt{\%d} outputs a number in denary
format, whereas \texttt{\%x} outputs in hex. The way this works is that the
variables to be printed are placed into the function stack, and then
whenever a token is encountered, it simply replaces it with whatever
is at the current stack pointer, and then moving the stack pointer so
that it is now at the next variable.

\subsection{\texttt{\%n}}
\label{sec:org2d92dc5}

\texttt{\%n} is a special formatting character which, instead reading the
value stored at the stack pointer, will instead overwrite it with the
number of bytes read in so far.

\section{Attacks}
\label{sec:org9ee0fa2}

\subsection{Denial of Service (DoS)}
\label{sec:org4e51a0c}

\texttt{\%s} is interpreted by \texttt{printf()} as a command which will use the
current value in function stack as a pointer to a null-terminated
string. This means that the program will unconditionally dereference a
value, this leads to 2 different situations:

\begin{enumerate}
\item The address is valid and accessible to the program, in which
\end{enumerate}
case it will continually print the characters found at that address
until the null-character (\texttt{\textbackslash{}0}) is met.
\begin{enumerate}
\item The address is invalid/out-of-bounds to the current program,
\end{enumerate}
resulting in a segmentation-fault (seg-fault), which causes the OS to
terminate execution.

Denial of Service works by entering enough \texttt{\%s} characters until the
value in the stack is interpreted as an invalid pointer.

An example input to cause the second situation is \texttt{\%s\%s\%s\%s\%s\%s\%s\%s\%s}
This continually moves the stack pointer, dereferencing the values and
printing what is stored in those addresses.

Another DoS attack can be done by what is known as "Stack Smashing"
which is when a large number of formatting characters are input
(eg. \texttt{\%d}, \texttt{\%x}, \texttt{\%c}, etc.), if enough of these formatting characters are
entered, the stack pointer will move out of the function call stack.

\subsection{Reading Values on the Stack}
\label{sec:orgfc65fb7}

When \texttt{printf()} is called, it loads the values to be printed into its
call stack. These values are then read off every time the string
formatting characters are met. It is possible to read values which
aren't in the \texttt{printf()} call stack by moving the stack pointer outside
of the scope of the function by continually entering the format
characters. This then allows access to variables stored by the
program.

\subsubsection{Variables vs Pointers}
\label{sec:org861a0a3}

Some variables are statically allocated in the program stack at
compile time (eg \texttt{char[6]}), these values can be directly
output by using specific formatting characters (\texttt{\%x}).

Other variables are dynamically allocated (\texttt{char*}) in the program heap, and
can only be accessed by dereferencing a pointer to them. To print
them, specific formatting characters (\texttt{\%s}) must be used as they interpret
the value on the stack as an address rather than a value.

\subsubsection{Data Structures}
\label{sec:org21da245}

Data Structures are typically dynamically allocated, as a result the
pointer to them will only point to the first value stored. In order to
print off the whole structure, we can utilise the fact that data is
stored in contiguous blocks of memory. This means that if an attacker
knows the address of the first element, it is possible to calculate
the addresses of further variables by noting the type of the data
structure (eg int, char, bool, etc) and using the size of the type as
the size to step through in memory.

\subsection{Writing Values to the Stack}
\label{sec:orge4c0fd5}

This attack utilises the \texttt{\%n} operator described earlier. If an
attacker wants to change the value stored in a particular location in
memory, then the address must first be found in the program stack,
then \texttt{\%n} can be used to change the value to be the number of bytes
read by \texttt{printf()} so far. This attack requires the address of the
variable to be changed.

\section{Lab 1: String Format Attack}
\label{sec:org3a0e064}

\subsection{Task 1.i: Crash the Program}
\label{sec:orgd3794b2}

The program can be crashed by entering the string \texttt{\%s\%s\%s\%s\%s\%s\%s\%s\%s}


\subsection{Task 1.ii: Print out the Value of \texttt{secret[1]}}
\label{sec:org8341271}

The address of \texttt{secret[1]} is printed out by the source program. As a
result we know the exact address to access. The program allows for 2
values to be input, one is a number, the other is a string. By putting
the decimal encoded address of \texttt{secret[1]} as the number to be input, it
gets placed in the stack in some location. This value can then be
found and accessed by entering special \texttt{printf()} formatting
characters, e.g. \texttt{\%x} which prints out values as a hex-encoded
number. Once the address that was placed earlier has been found in the
resulting output string, the attacker can then replace the
corresponding \texttt{\%x} character with \texttt{\%s} which then prints out the value
stored in that address (which in this case is \texttt{secret[1]}).


\subsection{Task 1.iii: Modify the value of \texttt{secret[1]}}
\label{sec:org5fe8e5b}

By accessing the value of \texttt{secret[1]} as explained in \textbf{Task 1.ii},
instead of simply printing the value of \texttt{secret[1]} with \texttt{\%s}, it is
possible to change the value stored by instead using \texttt{\%n} which
replaced the value stored by the number of bytes printed by \texttt{printf()} so
far.


\subsection{Task 1.iv: Modify the value of \texttt{secret[1]} to a chosen value}
\label{sec:orgfed217e}

From the previous task, it is possible for the attacker to modify the
value of \texttt{secret[1]}. From this point, it is possible to modify
\texttt{secret[1]} by using the same method of attack from \textbf{Task 1.iii}, but
before utilising \texttt{\%n} to change the value, an arbitrary number of
characters (e.g. \emph{1}) can be entered which will increment the value
updated by \texttt{\%n} by 1 for each character printed.

\subsection{Task 2.i: Cause the program to Crash}
\label{sec:org76e8d41}

This can be done in exactly the same way as \textbf{Task 1.i}, by entering
the string \texttt{\%s\%s\%s\%s\%s\%s\%s\%s\%s}.


\subsection{Task 2.ii: Print out the Value of \texttt{secret[1]}}
\label{sec:org4c681ae}

The modified program no longer allows the attacker to enter the
address of \texttt{secret[1]} directly. As a result, the attacker needs to
enter the address and access that address in a single string.

Due to the fact that Virtual Address Randomisation has been turned
off, it means that during every instance of the attack, the address of
\texttt{secret[1]} remains the same. Using this knowledge, it is possible to
input that address as a ascii-encoded bytes which can be passed into
the program from a file. This address can then be accessed in a
similar way to the previous attack by repeatedly entering \texttt{\%x} in the
ascii-encoded file until that address is located. Finally, the
attacker can replace the \texttt{\%x} corresponding to the address of
\texttt{secret[1]} with a \texttt{\%s} which will print out the value stored there.


\subsection{Task 2.iii: Modify the Value of \texttt{secret[1]}}
\label{sec:org5c874f2}

By utilising the steps taken in \textbf{Task 2.ii}, it is possible to modify
the value stored in \texttt{secret[1]} by replacing the \texttt{\%s} which prints the
value stored in \texttt{secret[1]} with \texttt{\%n} which will modify the value.


\subsection{Task 2.iv: Modify the Value of \texttt{secret[1]} to a chosen value}
\label{sec:org726fbf2}

Similarly to tasks: \textbf{2.iii} and \textbf{1.iv}, it is possible to modify the
value in \texttt{secret[1]} to a chosen value by first being able to modify
the value in general, and then utilising the fact that \texttt{\%n} increases
for every byte printed so far. From this, it means that it is possible
for an attacker to generate an ascii-encoded file with the number of
bytes desired to result in modifying \texttt{secret[1]} to the desired value.

\section{Possibilities and Limitations}
\label{sec:org0b97fc7}

Using this attack, attackers can do the following:

\begin{itemize}
\item Overwrite important program flags that control access privileges: if
\end{itemize}
the program  you're running has Set-UID privilages, you can leverage
the permissions  of the privilaged program to write higher privilages
to a malicious program.

\begin{itemize}
\item Overwrite return addresses on the stack, function pointers, etc
Writing any value, as demonstrated above, is possible through the
use of dummy output characters. To write a value of 1000, a simple
padding of 1000 dummy characters would do for example.
\end{itemize}

\section{History}
\label{sec:org6f9fb25}

String formatting to change behaviour has been known since 1989, but
as an attack vector, celebrates it's 20th year anniversary this
year. The attack itself was uncovered while running a security audit
on the "ProFTPD" program: essentially a feature rich FTP server,
during which it was shown that by passing certain values into an
unguarded \texttt{printf()} function, privilage escalation could be
achieved. Since this was uncovered, a number of prevention mechanisms
have been devised to assist programmer's in detecting the vulnerability.

\section{Prevention}
\label{sec:orgcc44a07}

\begin{itemize}
\item Compiler Support
\end{itemize}
\begin{itemize}
\item Address randomization:  just like the countermeasures used to protect
against buffer-overflow attacks, address randomization makes it
difficult for the attackers to find out what address they want to read/write
\end{itemize}

\section{Detection}
\label{sec:orge26ed67}

\begin{itemize}
\item x86 compiled binary analysis
\end{itemize}
\end{document}
