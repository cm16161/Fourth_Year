% Created 2019-10-31 Thu 13:09
% Intended LaTeX compiler: pdflatex
\documentclass[11pt]{article}
\usepackage[utf8]{inputenc}
\usepackage[T1]{fontenc}
\usepackage{graphicx}
\usepackage{grffile}
\usepackage{longtable}
\usepackage{wrapfig}
\usepackage{rotating}
\usepackage[normalem]{ulem}
\usepackage{amsmath}
\usepackage{textcomp}
\usepackage{amssymb}
\usepackage{capt-of}
\usepackage{hyperref}
\author{Chetan Mistry}
\date{\today}
\title{String Format Attack}
\hypersetup{
 pdfauthor={Chetan Mistry},
 pdftitle={String Format Attack},
 pdfkeywords={},
 pdfsubject={},
 pdfcreator={Emacs 26.3 (Org mode 9.1.9)},
 pdflang={English}}
\begin{document}

\maketitle
\tableofcontents


\section{Abstract}
\label{sec:orgc210d99}

A string format attack works by exploiting vulnerabilities in source
code to examine and/or modify the values stored in variables.


\section{String Formatting in C}
\label{sec:org0f1d18c}

Some String manipulation functions in C have vulnerabilities which can
be exploited by an attacker. An example of this is the function
"scanf" which allows the user to provide an arbitrary input.

\subsection{scanf()}
\label{sec:org65356e6}

"scanf()" is a function which reads from stdin and stores it into a
variable. This input can be anything (strings, numbers, etc), and can
have any length.

\subsection{printf()}
\label{sec:org8ab0242}

"printf()" is a function which takes in a string and any parameters to
that string (e.g variables) and will output to stdout. The variables
can be printed by using special tokens in the string which will map to
specific output formats, for example "\%d" outputs a number in denary
format, whereas "\%x" outputs in hex. The way this works is that the
variables to be printed are placed into the function stack, and then
whenever a token is encountered, it simply replaces it with whatever
is at the current stack pointer, and then moving the stack pointer so
that it is now at the next variable.

\subsection{"\%n"}
\label{sec:org8ba7e0d}

"\%n" is a special formatting character which, instead reading the
value stored at the stack pointer, will instead overwrite it with the
number of bytes read in so far.

\section{Attacks}
\label{sec:orgdff4b40}

\subsection{Denial of Service (DoS)}
\label{sec:org5581dcf}

"\%s" is interpreted by printf() as a command which will use the
current value in function stack as a pointer to a null-terminated
string. This means that the program will unconditionally dereference a
value, this leads to 2 different situations:

\begin{enumerate}
\item The address is valid and accessible to the program, in which
\end{enumerate}
case it will continually print the characters found at that address
until the null-character ($\backslash$0) is met.
\begin{enumerate}
\item The address is invalid/out-of-bounds to the current program,
\end{enumerate}
resulting in a segmentation-fault (seg-fault), which causes the OS to
terminate execution.

Denial of Service works by entering enough "\%s" characters until the
value in the stack is interpreted as an invalid pointer.

An example input to cause the second situation is "\%s\%s\%s\%s\%s\%s\%s\%s\%s"
This continually moves the stack pointer, dereferencing the values and
printing what is stored in those addresses.

Another DoS attack can be done by what is known as "Stack Smashing"
which is when a large number of formatting characters are input
(eg. \%d, \%x, \%c, etc.), if enough of these formatting characters are
entered, the stack pointer will move out of the function call stack.

\subsection{Reading Values on the Stack}
\label{sec:orgbbcfd68}

When printf() is called, it loads the values to be printed into its
call stack. These values are then read off every time the string
formatting characters are met. It is possible to read values which
aren't in the printf() call stack by moving the stack pointer outside
of the scope of the function by continually entering the format
characters. This then allows access to variables stored by the
program.

\subsubsection{Variables vs Pointers}
\label{sec:org66c4b54}

Some variables are statically allocated in the program stack at
compile time (eg char[6]), these values can be directly
output by using specific formatting characters (\%x).

Other variables are dynamically allocated (char*) in the program heap, and
can only be accessed by dereferencing a pointer to them. To print
them, specific formatting characters (\%s) must be used as they interpret
the value on the stack as an address rather than a value.

\subsubsection{Data Structures}
\label{sec:orga0f0116}

Data Structures are typically dynamically allocated, as a result the
pointer to them will only point to the first value stored. In order to
print off the whole structure, we can utilise the fact that data is
stored in contiguous blocks of memory. This means that if an attacker
knows the address of the first element, it is possible to calculate
the addresses of further variables by noting the type of the data
structure (eg int, char, bool, etc) and using the size of the type as
the size to step through in memory.

\subsection{Writing Values to the Stack}
\label{sec:org0988e51}

This attack utilises the "\%n" operator described earlier. If an
attacker wants to change the value stored in a particular location in
memory, then the address must first be found in the program stack,
then "\%n" can be used to change the value.
\end{document}
